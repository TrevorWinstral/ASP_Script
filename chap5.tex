\chapter{General Poisson Point Processes}
\textbf{Reference} Lectures on the Poisson Process (Penrose), Poisson Processes (Kingman)

\section{Introduction}
\textbf{Question} How can we represent points on $\mathbb{R}_+$ mathematically?
\begin{enumerate}
	\item A set of points $\mathcal{S}=\{S_1, S_2,...\}$ \label{q:3}
	\item 'Time point of view', ie $T_1,T_2,...$ where $T_i$ = time between the $(i-1) $'th and $i $'th point. \label{q:1}
	\item Cadlag formulation with values in $ \mathbb{N}$. $N_t=$ number of points in $[0,t]$. \label{q:2}
	\item Measure $N: \mathcal{B}(\mathbb{R}_+) \to \mathbb{N}$ with $N(A)=$ number of points in $A$. \label{q:4}
\end{enumerate}
\textbf{Goal} Define $\Omega \to $'set of points'. For a general state space $\mathbb{R}^2, [0,1]^2,$ a manifold, etc. 
\ref{q:1} and \ref{q:2} are specific to $\mathbb{R}_+$, so they do not generalize. \ref{q:3} is not very easy to describe. \ref{q:4} is actually nice, so we will  use this point of view.

\textbf{Framework} $(E,d)$ a Polish space (separable, complete, metric space). $\mathcal{E}$ Borel $\sigma$-algebra. $\mu: \sigma$ finite measure on $(E, \mathcal{E})$, ie  $\exists  B_i \uparrow E: \mu(B_i)<\infty$ where $B_i \uparrow E \iff B_1 \subset B_2 \subset...: \bigcup_{i\geq 1}B_i=E$.

\begin{ex}[] Of such spaces: 
\begin{enumerate}
	\item $E=\{0\}, \mu = \delta_0$
	\item $E=\mathbb{R}_+, \mu = \lambda \mathcal{L}$
	\item $E=\mathbb{R}^2, \mu(dx)=\frac{1}{\pi} e^{- |x|^2}dx$ 'Gaussian'
\end{enumerate}
\end{ex}

\textbf{Goal} We wish to define a point process on $(E, \mathcal{E})$ where the 'number of points around $x$ ' $\approx \mu(dx)$ on $\mathbb{R}_+$. 

\section{Point Processes}
\textbf{Notation} $\mathcal{N}=\{\nu: \nu=\sigma-$finite measure st $\forall B \in \mathcal{E}: \nu(B) \in \mathbb{N} \cup \{+\infty\}\}$.
\noindent
\textbf{Measure Structure} Let $\mathcal{B}(\mathcal{N})$ be the $\sigma$-algebra generated by the sets $\{\nu \in \mathcal{N}: \nu(B)=k\}=\mathcal{N}_k$ for $B \subset E$ meas and $k \in \mathbb{N}$. $\to (\mathcal{N}, \mathcal{B}(\mathcal{N}))$ measured space.

\begin{prop}[]
	Let $\mathcal{N}_{<\infty}=\{\nu \in \mathcal{N}: \nu(E)<\infty \}$, there exists meas maps $\tau: \mathcal{N}_{< \infty} \to \mathbb{N}, X_i: \mathcal{N}_{< \infty} \to E$ st $\forall \nu \in \mathcal{N}_{<\infty}: \nu = \sum_{i=0}^{\tau(\nu)} \delta_{X_i(\nu)}$.
\end{prop}

\begin{defn}
	A point process on $(E, \mathcal{E})$ is a RV $N$ with values in $ \mathcal{N}$. '$N$ is a random $\sigma$-finite measure', $N \leftrightarrow$ 'random set of points'. 
\end{defn}
This means $N: \Omega \to \mathcal{N} $ meas, for any fixed $B \subset E: N(B): \Omega \to \mathbb{N}\cup \{+\infty\}$ is measurable. 
Thus a stochastic process corresponds to $(N(B))_{B \in \mathcal{E}}$. '$N(B)=$ number of points in B'.

\begin{ex}[] Point Processes:
\begin{itemize}
	\item $N=0$ a.s.  $\to$ empty set
	\item $E=[0,1], X$ RV on  $[0,1]$.  $N=\delta_X$ is a point process.
	\item $X_1,...X_n$ iid RV on $[0,1]$,  $N=\delta_{X_1}+...+\delta_{X_n}$ is a point process.
\end{itemize}
\end{ex}

\section{Poisson Point Processes}
\textbf{Setup} $(E, \mathcal{E})$ Polish, $\mu$ fixed $\sigma$-finite measure (think of $\lambda \mathcal{L}$ ), $ \mathcal{N}=\{\sigma$ finite counting measure$\}$, $(\Omega, F, \mathbb{P} )$ abstract prob space.

\begin{defn}
	A Poisson process with intensity $\mu$ on $(E, \mathcal{E})$ ($ppp(\mu)$) is a point process st:
 \begin{enumerate}
	 \item $\forall B_1...B_k \subset E$ meas and disjoint: $N(B_1)...N(B_k)$ are indep.
	 \item $\forall B \subset E$ meas $N(B) \sim Pois(\mu(B))$.
\end{enumerate}
\end{defn}

\section{Existence and Uniqueness}
\textbf{Question} Does there always exist a $ppp(\mu)$ on $E$?
\subsubsection{Spaces with finite measure}
\begin{prop}[]
	Let $Z \sim Pois(\mu(E))$, $(X_i)_{i\geq 1}$ iid where $X_i \sim \frac{\mu(.)}{\mu(E)}$. Then $N= \sum_{i=1}^{Z} \delta_{X_i} $ is a $ppp(\mu)$ on $E$.
\end{prop}

\subsubsection{Superposition}
\begin{lemma}[]
	Let $\lambda = \sum_{i=1}^{\infty} \lambda_i, \lambda_i\geq 0$. $X_i \sim Pois(\lambda_i), i \geq 1$ indep, then $X = \sum_{i=1}^{\infty} X_i \sim Pois(\lambda)$.
\end{lemma}

\begin{theorem}[]
	Let $N_i, i\geq 1$ be a sequence of indep $ppp(\mu_i)$ where $\mu_i$ and $\mu = \sum_{i=1}^{\infty} \mu_i$ are $\sigma$-finite measures. Then $N= \sum_{i=1}^{\infty} N_i$ is a $ppp(\mu )$.
\end{theorem}

\begin{cor}[]
	$\mu \ \sigma$-finite measure on $(E, \mathcal{E})$, then $\exists\ ppp(\mu)$ on $E$.
\end{cor}

\subsubsection{Uniqueness}
Let $N$ be a $ppp(\mu )$ on $E$, define $P_N = $ law of $N$ ($\rightarrow$ a probability meas on $ \mathcal{N}$).

\begin{prop}[]
	Let $N, N'$ be two $ppp(\mu )$ on $(E, \mathcal{E})$ then $P_N = P_{N'}$.
\end{prop}

\begin{theorem}[Representation of ppp as Proper Processes]
	Let $N$ be a $ppp(\mu )$ on $(E, \mathcal{E})$, there exists some RV $\tau \in \mathbb{N} \cup \{+\infty \}$ st: $X_n \in E, n\geq 1: N = \sum_{i=1}^{\tau(} $
\end{theorem}

\section{Laplace Functional}
$N$ a random meas on $(E, \mathcal{E})$ for $u:E \to \mathbb{R}$ what should we interpret $\int_E u dN$ as?

\begin{lemma}[]
	$X \sim Pois(\lambda ), \lambda > 0$, then $\forall u\geq 0: \mathbb{E}_{} \left[ e^{-u X} \right] = exp( - \lambda (1 - e^{-u}))$.
\end{lemma}

\begin{defn}
	Let $N$ be a point process on $(E, \mathcal{E})$, for every $u:E\to \mathbb{R}_+$ define $L_N(u) = \mathbb{E}_{} \left[ exp(- \int u(x) N(dx) \right] $
\end{defn}

\begin{rmk}[]
	$\int_E u(x) N(dx) = \int_E u dN$ is a RV.
\end{rmk}

\begin{theorem}[Characterization via Laplace Functional]
	Let $\mu \ \sigma$-finite meas on $(E, \mathcal{E})$. Let $N$ be a point process on $E$. TFAE:
\begin{enumerate}
	\item $N$ is a $ppp(\mu)$ 
	\item $\forall u:E \to \mathbb{R}_+$ meas: $L_N(u) = exp(- \int_E 1- e^{-u(x)} \mu (dx))$
\end{enumerate}

\end{theorem}

\section{Simple Processes}
\begin{rmk}[]
	For $x \in E,\ \{x\}$ is meas. because $E$ is Polish.
\end{rmk}

\begin{defn}
	A measure $\eta \in \mathcal{N}$ is said to be simple if $\forall x \in E: \eta(\{x\}) \leq 1$.
\end{defn}

\begin{prop}[]
	$\{\eta: \eta$ is simple$ \}$ is measurable in $ \mathcal{N} $.
\end{prop}

\begin{theorem}[]
	Assume that $\mu $ is a diffuse ($\forall x: \mu (\{x\}=0$) $\sigma$ finite measure. Then every $ppp( \mu )$ is simple a.s.
\end{theorem}

\textbf{Consequence} $\exists \tau,\ X_i$ RV, $X_i \neq X_j$ if $i \neq j$ a.s.: $N=\sum_{i=1}^{\tau} \delta_{x_i}$ a.s.

\section{Mapping and Restriction}
$(E, \mathcal{E}), (F, \mathcal{F})$ Polish spaces, $\mu\ \sigma$-finite measure on $E$,  $T:E \to F$ meas, $T\#\mu $ push forward measure of $\mu $ under $T$ [$T\#\mu(B)=\mu(T^{-1}(B))$].

\begin{theorem}[]
	Assume that $T\#\mu$ is $\sigma$-finite. Let $N$ be a $ppp(\mu)$ on $E$, then $T\#N$ is a $ppp(T\#\mu)$ on $F$.
\end{theorem}

\begin{ex}[]
	$E=\mathbb{R},\ F=\mathbb{Z},\ T:E \to F; x \to \lfloor x \rfloor,\ \mu= \mathcal{L},\ T\#\mu=|.|$.
\end{ex}
\noindent
\textbf{Notation} If $\nu $ is a measure on $E$, $C \subset E$ meas.  $\nu _C: \nu(. \cap C)$

\begin{theorem}[Restriction]
	Let $C_1, C_2,... \subset E$ meas. and disjoint. If $N$ is a  $ppp(\mu)$ on  $E$, then $N_{C_1}, N_{C_2}...$ are indep $ppp$ with resp. intensities $\mu_{C_1}, \mu_{C_2},...$	
\end{theorem}

\section{Marking}
\noindent
\textbf{Motivation} Cars on a highway, at time 0 the position of the cars is a $ppp(1)$ on $\mathbb{R}$ (that means on average 1 car per kilometer of highway). We put an observer (Olga) at 0 on $\mathbb{R}$.

Case 1: All of the cars have speed 50km/h, we want to study $X=$ number of cars seen by Olga in 1 hour. What is the law of $X$? $X \sim Pois(50)$.

Case 2: The cars have a random speed $ \sim \mathcal{U}([50,100]) $. What is the law of $X$? It may at first seem complicated, but it is not!

\noindent
\textbf{Framework} $(E, \mathcal{E})$ Polish, $\mu=\sigma$-finite. $(F, \mathcal{F}, \nu )$ Polish, Probability space.
\begin{defn}
	Let $N=\sum_{i=1}^{\tau} \delta_{X_i}$ a $ppp(\mu)$ on $E$. $Y_i$ iid RV with law $\nu $ indep of $N$. The marked point process is the PP on $E \times F$ defined by $M=\sum_{i=1}^{\tau} \delta_{(X_i, Y_i)}$.
\end{defn}

\begin{rmk}[]
	$X_i$ corresponds to the position of the cars in Case 2, and $Y_i$ to their speeds.
\end{rmk}

\begin{theorem}[]
	The marked process is a $ppp(\mu \otimes \nu )$.
\end{theorem}

\noindent \textbf{Conclusion} The General PPP we have defined gives us a very general way to talk about a random processes on a large class of spaces (Polish), which fulfill a Markov-like property. This tool will allow us to make much stronger statements in more specific cases.

\section{Standard Poisson Process}
In discrete time processes $(X_n)_{n\in N}$, the law is characterised by the law of $(X_{n_1},..X_{n_k}; n_1...n_k \in \mathbb{N})$. In continuous time processes we have $(X_t)_{t\geq 0}$, we need to define $X_t:\forall t \in \mathbb{R}$ which is not countable.

\noindent \textbf{Outset} We would like to define a renewal process which also fulfills the Markov property, enabling us to not have. Furthermore we would like a simple continuous time process which is in some way a 'universal' stationary process on $\mathbb{R}_+ \to \mathbb{N}$ with independent increments and jumps of size 1. We would also like to see if any of the ideas from the previous chapter can be specified to this context.

\textbf{Applications} Queuing processes, insurance claims, compound Poisson process.

\textbf{Framework} $(\Omega, F, \mathbb{P})$ probability space, time space: $\mathbb{R}_{+}=[0,\infty)$ 

There are 2 points of view: random points on $\mathbb{R}_{+}$ (reminiscent of PPP) or continuous time stochastic process (renewal process).

\section{Exponential Random Variables}
\textbf{Note} We will use the 2nd point of view here.

\begin{defn}
	Let $\lambda> 0$, a real RV $T$ is exponential with parameter $\lambda$ (we write $T \sim Exp(\lambda)$) if it has density $f(t) = \lambda e ^{-\lambda t}\chi_{\{t\geq 0\}}$. $\iff \forall t\geq 0 \mathbb{P}_{} \left[ T>t \right] = e^{-\lambda t}$
\end{defn}

\begin{prop}[Memoryless Property]
	Let $\lambda > 0$ and $T \sim Exp(\lambda)$. Then  $\forall s,t\geq 0: \mathbb{P}_{} \left[ T>s+t | T>t \right] = \mathbb{P}_{} \left[ T>s \right] $
\end{prop}
\begin{prop}[Minimum of indep Exponentials]
	Let $n\geq 0, T_1...T_n$ indep with $T_i \sim Exp(\lambda_i), \lambda_i > 0$: 
\begin{itemize}
	\item  $min\{T_1...T_n\} \sim Exp(\lambda_1+...+\lambda_n)$
	\item $\mathbb{P}_{} \left[ T_1 = min\{T_1...T_n\} \right] = \frac{\lambda_1}{\lambda_1+...+\lambda_n}$
\end{itemize}

\end{prop}
 
\textbf{Reminder} $X$ a real RV with density $f$, $Y$ a RV with values in some measurable space $E$ indep of X. Then $\forall \phi:\mathbb{R} \times E \to \mathbb{R}$ meas + bdd we have: $\mathbb{E}_{} \left[ \phi(X,Y) \right] = \int_{0}^{\infty} \mathbb{E}_{} \left[ \phi(x,Y) \right] f(x) dx$ 

\begin{prop}[Sum of Exponentials]
	Let $\lambda > 0, n\geq 1$. Let $T_1...T_n$ be iid $Exp(\lambda)$ RVs. Then  $S_n = T_1+...+T_n$ is  $\Gamma(n, \lambda)$ distributed. ie $S_n$ is continuous with density $f_{S_n}(t)=\lambda e^{-\lambda t} \frac{(\lambda t)^{n-1}}{(n-1)!}$
\end{prop}

We can check that $\Gamma(1,t)=Exp(\lambda)$

\section{Definition of Poisson Processes}
\textbf{Setup}  $\lambda > 0, (T_i)_{i\geq 0}$ iid $Exp(\lambda ), S_n = T_1+...+T_n$

\begin{defn}
	The stochastic process $N=(N_t)_{t\geq 0}, N_t = \sum_{i=1}^{\infty} \chi_{S_i \leq t}$ is called the Poisson process with intensity $\lambda $ ($pp(\lambda)$). The RVs  $T_1,T_2,...$ are the inter-arrival times and  $S_1,S_2,...$ the arrival times/jump times.
\end{defn}
\noindent
\textbf{Elementary Properties}
\begin{itemize}
	\item The mapping $t \to N_t$ is a.s. right continuous, with values in $\mathbb{N}$
	\item For fixed $t\geq 0$ $N_t \sim Pois(\lambda t)$ ie $\mathbb{P}_{} \left[ N_t = n \right] = \frac{(\lambda t)^n}{n!}e^{- \lambda  t}$
\end{itemize}

\noindent
\textbf{Comment} "A property hold a.s." $\iff \exists$ meas set $A: \mathbb{P}_{} \left[ A \right] =1$ and $\forall \omega \in A$ the property holds. 


\section{Markov Property}
\begin{theorem}[Markov Property of N]
	Fix $t\geq 0$, the stochastic process $N^{(t)}=(N^{(t)}_{s})_{s \geq 0}$ defined by $N^{(t)}_s = N_{t+s}-N_{t}$ is a Poisson process, independent of $(N_u)_{0 \leq u \leq t}$.
\end{theorem}

\section{Stationary and Independent Increments}
\textbf{Motivation} We want to describe the law of $(N_{t_0},...,N_{t_k})$, the key here is that they are not totally independent. If we have 5 points at time $t_0$ then we know at time $t_1$ there will be at least 5 points. So we look at the law of $(N_{t_1}-N_{t_0},...,N_{t_k}-N_{t_{k-1}})$ ie the increments.

\begin{defn}
	A stochastic process $(X_t)_{t\geq 0}$ is said to have indep and stationary increments if 
\begin{itemize}
	\item $\forall k \geq 1, \forall 0=t_0 < ... < t_k: X_{t_1}-X_{t_0}, ..., X_{t_k}- X _{t_{k-1}}$ are indep
	\item $\forall  s<t, \forall  n \geq 0: X_t - X_s \stackrel{law}{=} X _{t+h}-X_{s+h}$
\end{itemize}

\end{defn}

\begin{theorem}[Marginals of Poisson Process]
We have the following:
\begin{enumerate}
	\item $\forall k \geq 1, \forall 0=t_0<...<t_k: N_{t_1}-N_{t_0},...,N_{t_k}- N_{t_{k-1}}$ are indep
	\item $\forall s \leq t: N_t - N_s \sim Pois(\lambda (t-s))$
\end{enumerate}
In particular $N=(N_t)_{t\geq 0}$ has indep and stationary increments.
	
\end{theorem}

We know the law of $(N_{t_1},...,N_{t_k}$ for every fixed $t_1...t_k$. 
\begin{align}
	\mathbb{P}_{} \left[ N_{t_1}=m_1...N_{t_k}=m_k \right] =& \mathbb{P}_{} \left[ N_{t_1}=m_1, N_{t_2}-N_{t_1}=m_2 - m_1,..., N_{t_k}-N_{t_{k-1}}=m_k - m_{k-1} \right] \nonumber \\ 
	=& \prod_{i=1}^{k}\frac{(\lambda (t_0 - t_{i-1}))^{m_i-m_{i-1}}}{m_i - m_{i-1}} e^{- \lambda (t_i - t_{i-1})}
\end{align}

\section{Finite Marginals Characterization}
\textbf{Motivation}  Let $(N_t)_{t\geq 0}$ a stochastic process. Does the last formula from above ensure that the process is $pp(\lambda)$? No, we can define $\tilde{N}_t =  \sum_{i \geq 1}^{} \chi_{S_i<t} $, we could also just change the value of the process as some random points, thus when we fix $t_1,...,t_k$ we have 0 probability to see these.

In order to get a characterization we need to add some regularity assumptions.

\begin{defn}
	Let $N=(N_t)_{t \geq 0}$ be a continuous stoch process with values in $\mathbb{R}$. We say that $N$ is a counting process if the following holds a.s.:
\begin{enumerate}
	\item $N_0 = 0$ a.s.
	\item  $t \to N_t$ is non decreasing, right continuous, with values in $\mathbb{N}$ \label{continCond}
\end{enumerate}
In this case, we can define the jump times by setting $S_1=min\{t: N_t >0\}$ and by induction  $S_{i+1}= min\{t \geq S_i: N_t > N_{S_i}\}$.
\end{defn}

\begin{ex}[]
	$pp(\lambda )$ is a counting process.
\end{ex}

\begin{rmk}[]
	The condition \ref{continCond} is almost sure in the following manner: $\exists A$ meas. with $\mathbb{P}_{} \left[ A \right] =1$ st $\forall \omega \in A: t \to N_t(w)$ is non decreasing, right continuous, with values in $\mathbb{N}$.
\end{rmk}

\begin{theorem}[]
	Let $\lambda> 0:$ Let $N$ be a counting process, the following are equivalent:
\begin{enumerate}
	\item $N$ is $pp(\lambda)$
	\item $\forall k \geq 1, \forall t_0 =0 < t_1 <...<t_k, \forall n_1,...,n_k \in \mathbb{N}:$ \\ $\mathbb{P}_{} \left[ N_{t_1}-N_{t_0}=n_1,...,N_{t_k}-N_{t_k-1}=n_k \right] = \prod_{i=1}^k \frac{(\lambda (t_i - t_{i-1}))^{n_i}}{n_i!} e^{-\lambda (t_i - t_{i-1})} $
\end{enumerate}

\end{theorem}

\begin{rmk}[]
	By def $N$ is a $pp(\lambda)$ $\iff $ $N$ is a counting process with jumps of size 1 a.s. and  $S_1,S_2-S_1,...$ are iid  $exp(\lambda)$.
\end{rmk}

\section{Microscopic Characterization}
\begin{theorem}[]
	Let $N$ be a counting process, let $\lambda> 0$. TFAE:
\begin{enumerate}
	\item $N$ is $pp(\lambda)$ 
	\item $N$ has indep and stationary increments and $\mathbb{P}_{} \left[ N_t =1 \right] = \lambda t + o(t)$ and $\mathbb{P}_{} \left[ N_t \geq 2 \right] = o(t)$
\end{enumerate}

\end{theorem}

\section{Properties of Poisson Process}

\begin{theorem}[Law of Large Numbers]
Let $N$ be a $pp(\lambda), \lambda > 0$, then: $lim_{t \to \infty} \frac{N_t}{t}=\lambda$.
\end{theorem}

\textbf{Motivation} If we want to specify (and remove) certain points, for instance if the PP is describing arrival times at a bakery then say we want to differentiate between customers who are younger than 45 and those who are older. If we just look at one of these groups, what type of process are they?

\begin{theorem}[Thinning]
	Let $(N_t)_{t\geq 0} \sim pp(\lambda)$ with jump times $(S_i)_{i\geq 0}$. Let $(X_i)_{i\geq 0}$ iid $Ber(p)$ indep of $N$ (this is the differentiation, called the marking of $N$). Define $N_t^1 = \sum_{i\geq 1}^{} \chi_{S_i \leq t, X_i = 1}$ and $N_t^0 = \sum_{i \geq 1}^{} \chi_{S_i \leq t, X_i = 0}$.
\\ \noindent	
	$(N_t^0)$ and  $(N_t^1)$ are indep Poisson processes with respective rates  $\lambda_0 = (1-p)\lambda, \lambda _1=p\lambda $.
\end{theorem}

Let $(N_t^0)$ and $(N_t^1)$ be indep Poisson processes with respective rates $\lambda_0> 0, \lambda_1> 0$. Let $N_t = N_t^0 + N_t^1$.
\begin{theorem}[]
$N_t$ is a counting process and we define for every $i $: $X_i = \mathbbm{1}_{\textrm{\{i'th jump of }N_t\textrm{ is a jumping time of }N_t^1\}}$. Then  $N_t$ is a $pp(\lambda_0 + \lambda_1)$ and  $(X_i)$ is a marking of  $N$ with $\forall i: \mathbb{P}_{} \left[ X_i=1 \right] = \frac{\lambda_1}{\lambda_0+\lambda_1}$.
\end{theorem}

\noindent \textbf{Conclusion} We successfully defined a renewal process with the Markov property, we also found that this object is also a PPP, thus giving us a process which has the asymptotic behavior (LLN, etc) from the renewal process perspective and getting the Strong and Weak Markov Property from the Poisson Point Process perspective. 


